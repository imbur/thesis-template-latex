% pattern colors
\definecolor{negcolor}{RGB}{223,49,35}
\definecolor{newcolor}{RGB}{43,183,47}
\definecolor{delcolor}{RGB}{59,66,161}

% listing colors
\definecolor{backgroundcolor}{HTML}{F5F5F5}
\definecolor{keywordcolor}{HTML}{295F94}
\definecolor{commentcolor}{HTML}{AD95AF}
\definecolor{stringcolor}{HTML}{317ECC}

\newcommand{\union}{\cup}
\newcommand{\intersection}{\cap}
\newcommand{\parentheses}[1]{\left(#1\right)}
\newcommand{\op}[2]{\mathrm{#1}\parentheses{#2}}
\newcommand{\naturaljoin}{\bowtie}
\newcommand{\antijoin}{\, \triangleright \,}
\newcommand{\iq}{\mbox{\textsc{IncQuery}}\xspace}
\newcommand{\iqpl}{\mbox{\textsc{IncQuery}} Pattern Language\xspace}
\newcommand{\iqd}{\mbox{\textsc{IncQuery-D}}\xspace}
\newcommand{\eiq}{\mbox{\textsc{EMF-IncQuery}}\xspace}
\newcommand{\emfapi}{\mbox{EMF API}\xspace}
\newcommand{\tb}{Train Benchmark\xspace}
\renewcommand{\ie}{i.e.,\@\xspace}
\renewcommand{\Ie}{I.e.,\@\xspace}
\renewcommand{\eg}{e.g.,\@\xspace}
\renewcommand{\Eg}{E.g.,\@\xspace}
\renewcommand{\etal}{et al.\@\xspace}
\renewcommand{\etc}{etc.\@\xspace}

% scenarios
\newcommand{\scenario}[1]{\textsf{#1}\xspace}
\newcommand{\batch}{\scenario{Batch}}
\newcommand{\inject}{\scenario{Inject}}
\newcommand{\repair}{\scenario{Repair}}

% phases 
\newcommand{\phase}[1]{\textsf{#1}\xspace}
\renewcommand{\read}{\phase{read}}
\renewcommand{\check}{\phase{check}}
\newcommand{\transformation}{\phase{transformation}}
\newcommand{\recheck}{\phase{recheck}}

\newcommand{\readandcheck}{\phase{\read and \check}}
\newcommand{\transformationandrecheck}{\phase{\transformation and \recheck}}
% queries
\newcommand{\query}[1]{\textsf{#1}\xspace}
\newcommand{\connectedsegments}{\query{Connected\-Segments}}
\newcommand{\poslength}{\query{PosLength}}
\newcommand{\routesensor}{\query{Route\-Sensor}}
\newcommand{\semaphoreneighbor}{\query{Semaphore\-Neighbor}}
\newcommand{\switchsensor}{\query{Switch\-Sensor}}
\newcommand{\switchset}{\query{SwitchSet}}

% table headers
\newcommand*{\theadswfirst}[1]{\multicolumn{1}{|c|}{\bfseries \begin{sideways}#1\end{sideways}}}
\newcommand*{\theadsw}[1]{\multicolumn{1}{c|}{\bfseries \begin{sideways}#1\end{sideways}}}
\newcommand*{\theadfirst}[1]{\multicolumn{1}{|c|}{\bfseries #1}}
\newcommand*{\thead}[1]{\multicolumn{1}{c|}{\bfseries #1}}

% for the hyperref package
\renewcommand{\sectionautorefname}{Section}
\renewcommand{\subsectionautorefname}{Section}
\renewcommand{\subsubsectionautorefname}{Section}

\newcommand{\patternscale}{0.23}
\newcommand{\yedscale}{0.50}
\newcommand{\ecorescale}{0.50}

\newcommand{\yes}{$\CIRCLE$\xspace}
\newcommand{\somewhat}{$\LEFTcircle$\xspace}
\newcommand{\no}{$\Circle$\xspace}

\newcommand{\forallqueries}[1]{\foreach \q in {ConnectedSegments,PosLength,RouteSensor,SemaphoreNeighbor,SwitchSensor,SwitchSet}{#1}}

\newcommand{\benchmarkresult}[3]{
\begin{figure*}[p]
	\centering
	\includegraphics[width=\textwidth]{benchmark-results/#1-#2}
	\caption{#3.}
	\label{fig:#1-#2}
\end{figure*}}

\newcommand{\patterngraphics}[1]{
	\centering
	\includegraphics[scale=\patternscale]{pattern-#1}
 	\caption{The \textsf{#1} pattern.}
 	\label{fig:pattern-#1}
}

\newcommand{\tbpattern}[1]{
\begin{figure}[H]
	\patterngraphics{#1}
\end{figure}}

\newcommand{\yedfigure}[2]{
\begin{figure}[H]
	\centering
	\includegraphics[scale=\yedscale]{yed/#1}
	\caption{#2.}
	\label{fig:#1}
\end{figure}}

\newcommand{\yedfigurelarge}[2]{
\begin{figure*}[htbp]
	\centering
	\includegraphics[scale=\yedscale]{yed/#1}
	\caption{#2.}
	\label{fig:#1}
\end{figure*}}

\newcommand{\tbtransformation}[2]{
\begin{figure}[H] 
	\centering
	\includegraphics[scale=\patternscale]{transformation-#1-#2}
 	\caption{The \textsf{#1} transformation for \textsf{#2}.}
 	\label{fig:transformation-#1-#2}
\end{figure}}

\lstset{
	numbers=left,
	numberstyle=\scriptsize\ttfamily,
	stepnumber=1,
	numbersep=5pt,
	%
	basicstyle=\scriptsize\ttfamily,
	backgroundcolor=\color{backgroundcolor},
	keywordstyle=\color{keywordcolor}\bfseries,
	commentstyle=\color{commentcolor},
	stringstyle=\color{stringcolor},
	identifierstyle=, % nothing happens
	%
	showstringspaces=false, % no special string spaces
	aboveskip=3pt,
	belowskip=3pt,
	columns=flexible,
	keepspaces=true,
	breaklines=true,	
	frameround=tttt,
	captionpos=b,
	tabsize=2,
	frame=tb,
	framerule=0pt,
	framexleftmargin=0.25em,
}

% languages

\lstdefinelanguage{iqpl}
{
	morekeywords={@QueryBasedFeature,@Constraint,count,pattern,package,neg,find,import,true,false,or,check,job,action,state,severity,location,message,oclIsKindOf,self,exists,includes,invariant,class},
	sensitive=true,
	morecomment=[l]{//},
	morecomment=[s]{/*}{*/},
	morestring=[b]{"},
}

\lstdefinelanguage{xtend}{
	morekeywords={class,for,def,val,var,cached,case,default,extension,false,import,JAVA,WORKFLOWSLOT,let,new,null,public,protected,private,create,switch,this,true,reexport,around,if,then,else,context,DEFAULT_NO_UPDATE_AND_DISAPPEAR,DEFAULT,APPEARED,DISAPPEARED,UPDATED,processor},
	morecomment=[l]{//},
	morecomment=[s]{/*}{*/},
	morestring=[b]",
	mathescape=true,
}

\lstdefinelanguage{sparql}{
	morekeywords={SELECT, DISTINCT, WHERE, OPTIONAL, FILTER, NOT, EXISTS, sameTerm, bound},
}

% listings

\newcommand{\listingiqpl}[2]
{
	\lstset{
		language=iqpl
	}
	\lstinputlisting[label=lst:#1, caption=#2.]{code/#1.eiq}
}

\newcommand{\listingjava}[2]{
	\lstset{
		language=Java
	}
	\lstinputlisting[label=lst:#1, caption=#2.]{code/#1.java}
}

\newcommand{\listingsparql}[2]{
	\lstset{
		language=sparql
	}
	\lstinputlisting[label=lst:query-#1, caption=#2.]{queries/#1.sparql}
}

\lstdefinelanguage{turtle}
{
	morekeywords={@prefix,@base,@forSome,@forAll,@keywords},
	morecomment=[l]{\#\ },
	morestring=[b]\"
}


\lstdefinelanguage{XML}
{
	morestring=[b]",
	morecomment=[s]{?}{?},
	morecomment=[s]{!--}{--},
	morekeywords={xmlns,version,xsi,xmi,graphml,graph},
}

\newcommand{\catchproblem}{\texttt{catchProblemFinder}\xspace}
\newcommand{\handlervar}{\texttt{handlerVariable}\xspace}
\newcommand{\baseindex}{\mbox{EMF-IncQuery} Base Index\xspace}
\newcommand{\screenshotscale}{0.74}
\newcommand{\stackedlines}[2]{\begin{tabular}{@{}c@{}}#1 \\ #2\end{tabular}}
\newcommand{\eobject}{\texttt{EObject}\xspace}

\newcommand{\todo}[1]{\textcolor{red}{\LARGE{TODO: #1}}\xspace}%
\newcommand{\Modeldriven}{Model-driven\xspace}
\newcommand{\modeldriven}{model-driven\xspace}
\newcommand{\matlab}{\textsc{Matlab}\xspace}
\newcommand{\simulink}{\textsc{Simulink}\xspace}
\newcommand{\matlabsimulink}{\textsc{Matlab-Simulink}\xspace}

\newcommand{\functionCall}[1]{\kern-0.25em(#1)}